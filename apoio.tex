-- Questões de Avaliação Interpolação
https://www.ufrgs.br/reamat/CalculoNumerico/livro-py/i1-interpolacao_polinomial.html

********************** x ******************

-- Conversão Binária => Decimal e Decimal => Binária
https://pt.calcuworld.com/calculadoras-matematicas/calculadora-binaria/

-- Calculadora Binária
https://calcuonline.com/calculadoras/calculadora-binaria/


-- Calculador Octal
https://miniwebtool.com/octal-calculator/


-- Calculadora Hexa
https://www.calculator.net/hex-calculator.html/


-- Representação Binária e Complemento de 2 (Carry / Estouro de Bit)
https://es.planetcalc.com/747/

-- Complemento de 1 e 2
https://medium.com/@jeanfelipemartinsdacosta/sinal-magnitude-bit-de-sinal-complemento-de-1-e-complemento-de-2-a1d32045ab18

********************** x ******************

-- Exercícios - Questões de Trapézio
https://exercicios.brasilescola.uol.com.br/exercicios-matematica/exercicios-sobre-trapezio.htm#questao-2

-- Círculo
https://mundoeducacao.bol.uol.com.br/matematica/circulo-circunferencia.htm

C (comprimento) = 2πr, em que π é aproximadamente 3,14
A (área) = π.r2

EDO - Prod. Bruno (Pela Lista) Pelo portal em CN

- Soma Hexa
https://www.calculator.net/hex-calculator.html/
https://miniwebtool.com/hex-calculator/?number1=D898&operate=1&number2=87

-- Soma Octal
https://miniwebtool.com/octal-calculator/

https://calcuonline.com/calculadoras/calculadora-binaria/

https://www.tudoengcivil.com.br/search/label/C%C3%A1lculo%20Num%C3%A9rico

Aulas práticas / MatLab
https://www.scilab.org/
http://www.scilab.org/download/downthks?success=ok

-- Indicar para os alunos / Várias Conversões / Soma / Subtração
http://www.multicalculadora.com.br/

-- Indicar para os alunos / Conversão Ponto Flutuante
http://webcalc.com.br/Matematica/form/notacao_numerica
Não IEE 754 https://carlosrafaelgn.com.br/aula/flutuante.html
Não IEE 754 https://guiatech.net/ieee-754-conversao/

http://www.calculadoraonline.com.br/calculadoras
http://wims.unice.fr/wims/wims.cgi

-- Conversão - medidas
http://webcalc.com.br/Matematica/form/notacao\_numerica

-- Complemento de 2
http://carlosrafaelgn.com.br/Aula/Binario.html

-- Complemento de 2; Estouro de Bit (Carry)
-- Indicar para os alunos (Carry)
https://es.planetcalc.com/747/

https://pt.calcuworld.com/calculadoras-matematicas/calculadora-binaria/

https://pt.wikibooks.org/wiki/Eletr%C3%B4nica_Digital/N%C3%BAmeros_bin%C3%A1rios_negativos

Cursos de Matemática e Eletrônica
https://www.youtube.com/channel/UCgzw2Y33TkduRWvaTF0BTEg/playlists

Curso de Sistemas de Numeração
https://www.youtube.com/watch?v=w8WgryYN4Ec&list=PLIHE326iMCHAXyNnBetAdlhZGb44C-eCx

http://homes.dcc.ufba.br/~alirio/files/cn2012.1/Erros.pdf

http://www.ebah.com.br/content/ABAAABBYYAB/calculo-numerico-modelagem-matematica?part=2

http://www.ebah.com.br/content/ABAAABBYYAB/calculo-numerico-modelagem-matematica

CN
https://www.youtube.com/watch?v=qHNkK2GLToc&list=PLIHE326iMCHAKqaupMhTorHU4vd5rmrmi

-- Interpolação Polinomial Inversa
=> https://www.youtube.com/watch?v=oc348f3b9Zg

https://www.mesalva.com/ensino-superior/engenharia

EXERCÍCIOS EXTRA

- Método de Gauss (Escalonamento), Jordan, Fatoração LU

Sistema de Equações
2x + 3y - z = 5
4x + 4y - 3z = 3
2x -3y + z = -1

Resposta: x = 1; y = 2; z = 3

- Método Gauss Jacobi e Seidel - Iterativo
10x +2y + z = 7
x + 5y + z = -8
2x + 3y + 10z = 6

x0 = 0.5; y0= -1; z0 = 0.5; estimativas iniciais

S = {1, -2, 1}

- Mínimos Quadrados = g = Fi * Teta; Eq. Reta e Eq Parábola (Ajuste de Curva)
Teta = [Fi\^T * Fi]\^-1 * Fi\^T * g

- Método Interpolação Polinomial Lagrange
P(x) = f(x0) * Lo(x) + f(x1) * L1(x) + f(x2) * L2(x)
x       -1  0   2
f(x)    4   1   -1
P(x) = f(x0) * L0(x) + f(x1) * L1(x) + f(x2) * L2(x)
L0(x0) = (x-x1) (x-x2) / (x0 - x1) (x0-x2)
Resposta: P(x) = 4[(x\^2-2x)/3] + 1[(x\^2-x-2)/-2] + (-1)[(x\^2+x)/6]
P(1) = -2/3

- Método Interpolação Polinomial Newton
P(x) = f(x0) + (x-x0) * k1 + (x-x0) * (x-x1) * K2 + (x-x0) * ( x - x1) * (x - x2) * k3 + ( x-x0) * (x-x1) * (x-x2) x (x-x3) *k4
x       f(x)    k1
-1      1       1 - 1 / 0 - -1
0       1
1       0
2       -1
3       -2

- Método Interpolação Polinomial Linear Newton
P(x) = f(xo) + (x-x0) * [(f(x1) - f(x0)) / (x1-x0)]

x       2       2,5     4
f(x)    0,5     0,4     0,25

f(2,2) = ? então xk E [xo;x1]; xk E [2;2,5]
Resposta: P(2,2) = f(2,2) = 0,46

******************************************* 
Exercício Lagrange 2 Grau (https://www.youtube.com/watch?v=Y9sa2S-ljX8)
P(x) = f(x0)*Lo(x) + f(x1)*L1(x) + f(x2)*L2(x)

(-2; 2), (0;-2), (4;1)

Resposta: P(x) = 11x\^2\24 - 13x\12 - 2

Resolução: https://www.youtube.com/watch?v=uJOv3QSr_fo
1. Sistemas Lineares (1o Grau e 2 Grau)
2. Lagrange (1o Grau e 2 Grau)
3. Newton (1Grau e 2 Grau)

x       -1  0    2
f(x)     4  1   -1

P(x) = a0 + a1x + a2x\^2
P(x) = 1 - (7/3)x + (2/3) x\^2
P(1) = ?? -0,66667

*************************************************************************
Exercícios:

https://www.tudoengcivil.com.br

Plano de Ensino - Cálculo Numérico - C.H.: 60h; 20 aulas de 3h (20 dias)
20 dias (5 meses)

-- Aula: 1 (Apresentação, Plano de Curso, Introdução) + Sist. de Numeração-07/08/18

-- Aula: 2 + Exercícios - 14/08/18
1. Sistema de Numeração
1.1. Conversão de Base: Decimal, Binário, Octal, Hexadecimal
1.2. Operações Aritméticas Binária (Postar material didático no Github)

-- Aula: 3 + Exercícios - 21/08/18
2. Teoria dos Erros - Aritmética de Ponto Flutuante
2.1. Representação Numérica
2.2. Arredondamento e Truncamento
2.3. Overflow e Underflow - SPF (Sistema de Ponto Flutuante) com máquina que opera por arredondamento
2.4. Palavra de 16 bits

3. Teoria dos Erros - Continuação...
3.1. Tipos de Erros - Erro Absoluto e Relativo

-- Aula: 4 + Exercícios - 28/08/18
3.2. Máximo Erro Relativo de Arredondamento (Propagação de erro)
(Análise de Erros nas Operações Aritméticas de Ponto Flutuante)

4. Zeros de Função (Isolamento das Raízes)
4.1. Determinação do Intervalo que contém o zero da função
4.2. Refinamento das raízes estimadas
4.3. Método da Bissecção
4.4. Método da Falsa Posição
4.5. Método de Ponto Fixo
4.5.1. Método de Ponto Fixo: Convergência e Divergência

-- Aula: 5 + Exercícios - 04/09/18
4.6. Método Newton-Raphson
4.7. Método da Secante
4.8. Comparação entre os métodos
[???] 4.9. Estudo Especial de Equações Polinomiais

-- Aula: 6 + Exercícios - 11/09/18
5. Sistemas Lineares - Métodos Diretos
5.1. Método de Gauss - Escalonamento
5.2. Método da Pivotação
5.3. Método de Jordan

-- Aula: 07 - Revisão + Exercícios - 18/09/18

-- Aula: 08 - Revisão + Exercícios - 25/09/18
-- Aula: 09 - Revisão + Exercícios - 02/10/18
-- Aula: 10 - Dia Avaliação P1     - 09/10/18

-- Aula: 11 + Exercícios - 16/10/18
   Correção da P1
5. Sistemas Lineares - Métodos Diretos
5.3. Método da Fatoração LU

-- Sistemas Lineares - Métodos Iterativos
5.4. Método de Gauss-Jacobi
5.5. Método de Gauss-Seidel
5.6. Comparação Entre os Métodos
[Talvez] 5.7. Método dos Mínimos Quadrados: Polinômio do 1 e 2 Grau. Eq da Reta e Parábola

-- Aula: 12 + Exercícios - 23/10/18
Ajuste de Curvas
6. Interpolação Polinomial
6.1. Definição e Interpretação Geométrica. 
6.2. Interpolação polinomial (Linear e Quadrática)
6.3. Resolução de Sistema Linear
6.4. Forma Lagrange
6.5. Forma de Newton

-- Aula: 13 + Exercícios - 30/10/18
6.6. Interpolação Inversa: Definição; Interpolação inversa linear e quadrática; estimativa de erros em problemas de interpolação inversa; 
6.7 Fenômeno de Runge para pontos igualmente espaçados.
[Talvez] 6.8. Funções Spline em Interpolação 

-- Aula: 14 + Exercícios - 06/11/18
7. Integração Numérica
[Estudar] 7.1. Fórmulas de Newton-Cotes
7.2. Regra dos Trapézios Simples
7.3. Regra dos Trapézios Repetida
7.4. Regra 1/3 de Simpson Simples e Repetida
7.5. Regra de Simpson Simples e Repetida

-- Aula: 15 + Exercícios - 13/11/18
[Estudar] 7.6. Quadratura Gaussiana a dois pontos (n=2)
7.7. Comparação entre a quadratura gaussiana a dois pontos (n=2), estudo do erro por quadratura gaussiana e as fórmulas de Newton.

-- Aula: 16 + Exercícios - 20/11/18 
[???] 8. Soluções Numéricas e Equações EDO
8.1. Problema do Valor Inicial
8.2. Métodos de Passo Um (Ou Passo Simples)
8.3. Métodos de Passo Múltiplo
8.4. Solução de equações diferenciais de primeira ordem pelo método de Euler e por expansão em série de Taylor.

-- Aula: 17 + Revisão P2 + Exercícios - 27/11/18
-- Aula: 18 Avaliação P2 - 04/12/18
-- Aula 19 + Correção P2 + Revisão Geral para a Final - 11/12/18

-- Aula 20 - 18/12/18
9. Etapas do MASP: objetivos; identificação do problema - observação, análise, plano de ação, ação, verificação e padronização; recapitulacão de todo o processo de solução do problema para trabalho futuro.

10.1. Equações de Ordem Superior
10.2. Problema de Valor de Contorno - O método das diferenças finitas
10.3. Aproximações por diferença-quociente para derivadas de qualquer ordem; 
10.4. Transformações de equações diferenciais em problemas lineares. 

**************************************************

O cálculo numérico consiste de métodos numéricos para obtenção de soluções aproximadas de problemas: álgebra linear e não linear, cálculo diferencial, estatística, integral e outros modelos matemáticos.

O Cálculo Numérico consiste na obtenção de soluções aproximadas de problemas de Álgebra Linear e Não-Linear, Estatística e Análise de Dados, Cálculo Diferencial e Integral e outros métodos matemáticos, utilizando métodos numéricos. Com a popularização de computadores de baixo custo e de alta capacidade de processamento, praticamente todas as atividades de Engenharia tem feito uso cada vez mais intensivo dos métodos e técnicas computacionais na resolução de problemas reais, para os quais as soluções manuais são impraticáveis e/ou imprecisas. Desta forma, o uso do computador como ferramenta de trabalho de cálculo numérico requer o entendimento dos seus princípios de operação e de como eles interferem nos resultados obtidos. Geralmente, é aceito como verdade que computadores não erram e que são os usuários é que cometem enganos que levam ao mal funcionamento do computador. Na realidade, o computador, como dispositivo de cálculo numérico, “comete” erros devido às suas características intrínsecas e o papel do usuário é quantificar esses erros e encontrar formas de, se não eliminá-los, pelo menos minimizá-los.

Exercício: 14/08/2018

 9 = (1001)2; 3 = (11)2; 6 = (110)2
+9 -3 =  (110)2; 
-9 +3 = -(110)2;  (1010)  em complemento de 2; (11111010) 8 bits
+9 +3 =  (1100)2;
-9 -3 = -(1100)2; (10100) em complemento de 2; (11110100) 8 bits

+9 -6 =  (11)2;
-9 +6 = -(11)2; (1101) em complemento de 2; (11111101) 8 bits
+9 +6 =  (1111)2;
-9 -6 = -(1111)2; (10001) em complemento de 2; (11110001) 8 bits

 55 = (110111); 32 = (100000)
 55 - 32 = 23 => (010111)2
-55 + 32 = (101001) em Complemento de 2; (11101001) 8 bits

 23 = (10111); 11 = (1011)
 23 - 11 = 12 => (1100)2
-23 + 11 = (110100) em complemento de 2; (11110100) 8 bits

Exercícios:
Ex1: (AC + BD + 85)B16 = 1EE B16; 494 Base 10; 0001 1111 1111 Base 2
Ex2: (DF + A5 + EB + 17)B16= 286 B16; 646 Base 10; 0010 1000 0110 Base 2

(89 + 78) Base8 = (Erro, não existe na base octal)
Ex1: (46 + 76 + 47) B8 = 213 Base 8; 139 Base 10; 010 001 011 Base 2
Ex2: (45 + 37 + 26 + 77) B8 = 231 Base 8; 153 Base 10; 010 011 001 Base 2


Erro Relativo e Erro Absoluto
Queremos saber quanto é a estimativa de distância de uma árvore para outra, seguindo passos largos podemos perceber que a distância entre elas é de 18 metros– esse é o valor experimental.

Em seguida, você realiza uma medição com uma fita métrica, mede a mesma distância e descobre que, na verdade, elas estão a 20 metros de distância uma da outra. Esse é o valor real.

O erro absoluto é 20 – 18 = 2 metros.

Agora vamos descobrir o erro relativo, primeiro dividimos o erro absoluto pelo valor obtido, o resultado iremos multiplicar por 100 para obter uma porcentagem, vejamos a seguir:

2÷20= 0,1 x 100= 10% erro relativo

Então, nosso erro relativo será 10% do valor real.

Exercício (Soma Hexadecimal)
http://www.multicalculadora.com.br/converter-hexadecimal-para-decimal/
C1D = 3101
2B8 = 696
3FF = 1023
Resultado: 12D4 (4.820)

27/08/2018
Corrigir Aula Revisão + Exercício letra 

v = 40320 e v* = 40320,042
Resposta: 4,2 * 10\^ -2 e 7.346 * 10 \^ -6


